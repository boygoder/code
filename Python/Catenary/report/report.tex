\documentclass{article}
\usepackage{ctex}
\usepackage{float}
\usepackage{lmodern}
\usepackage{amssymb,amsmath}
\usepackage{xcolor}
\usepackage{graphicx}

\title{悬链线拟合}
\author{}
\date{}



\begin{document}
\maketitle



\section{问题介绍}

作为世界上吞吐量最大的港口之一的宁波港,每年停靠在这个港口的船舶不计其数。目前需要人工巡检手段去监控这些船舶是否稳定的停靠在码头内,效率低下,也无法在较长的船舶停靠区内做到实时监测到揽绳的实时状态。

船舶系泊的揽绳是否松动是衡量船舶停靠的核心依据,我们希望通过无人化手段去监测缆绳是否松动。

对于如下图所示的二值图,我们要解决的问题是,细分出图片中每一根缆绳,拟合缆绳曲线,进而计算出缆绳的弯曲度。

\begin{figure}[H]
\begin{minipage}{0.48\linewidth}
  \centerline{\includegraphics[width = \textwidth]{./fig/example1.png}}
\end{minipage}
\hfill
\begin{minipage}{0.48\linewidth}
  \centerline{\includegraphics[width = \textwidth]{./fig/example2.png}}
\end{minipage}
\end{figure}


\section{解决思路}

数据: 二值图的像素矩阵,白色像素处值为1,黑色像素处值为0。

方程:悬链线是一种曲线。一条理想的链条悬挂在两个支撑点之间,并受到均匀的重力作用,形成悬链线的形状。理想的链条是可以完美弯曲、不能拉伸并且整个密度相同的链条。支撑点可以处于不同的高度,形状仍然是悬链线。一般的悬链线方程为\(y = a \cosh((x-b)/a) + c\)。

\begin{itemize}
\item
  单侧边缘提取:以水平方向为x轴,对于同一x,每条缆绳覆盖了多个像素点。通过单侧的边缘提取,每条缆绳的像素宽度变为1。\\
  \begin{figure}[H]
  \begin{minipage}{0.5\linewidth}
  \centerline{\includegraphics[width = \textwidth]{./fig/step1-1.png}}
  \end{minipage}
  \hfill
  \begin{minipage}{0.5\linewidth}
  \centerline{\includegraphics[width = \textwidth]{./fig/step1-2.png}}
  \end{minipage}
  \end{figure}

\item
  判定缆绳数量:对经过边缘提取后的像素矩阵,每一列求和,并消除可能出现的干扰点影响,最终求和结果的最大值就是图片中的缆绳数量。\\
  \begin{figure}[H]
  \begin{minipage}{0.5\linewidth}
  \centerline{\includegraphics[width = 0.8\textwidth]{./fig/step2-1.png}}
  \end{minipage}
  \hfill
  \begin{minipage}{0.5\linewidth}
  \centerline{\includegraphics[width = 0.8\textwidth]{./fig/step2-2.png}}
  \end{minipage}
  \end{figure}

\item
  区分不同缆绳:首先取出一段不同缆绳的无交点部分作为样本点,进行第一次曲线拟合。\\
  \begin{figure}[H]
  \begin{minipage}{0.5\linewidth}
  \centerline{\includegraphics[width = \textwidth]{./fig/step3-1.png}}
  \end{minipage}
  \hfill
  \begin{minipage}{0.5\linewidth}
  \centerline{\includegraphics[width = \textwidth]{./fig/step3-2.png}}
  \end{minipage}
  \end{figure}

\item
  曲线拟合:为了获取周围的样本点,根据拟合出的近似曲线,依次计算样本点和每一条曲线的距离。根据距离大小决定样本点应该归属的曲线,进行下一次曲线拟合。重复,直至获取所有的样本点。\\
  \begin{figure}[H]
  \begin{minipage}{0.5\linewidth}
  \centerline{\includegraphics[width = \textwidth]{./fig/step4-1.png}}
  \end{minipage}
  \hfill
  \begin{minipage}{0.5\linewidth}
  \centerline{\includegraphics[width = \textwidth]{./fig/step4-2.png}}
  \end{minipage}
  \end{figure}

\item
  计算弧度:弧度定义:每条曲线2个顶点连成一条直线,直线长度L,计算曲线上每个点和直线的距离,最大距离为H,弧度定义为H/L。这里两个顶点取为拟合曲线所用到的最左侧和最右侧样本点。\\
  \begin{figure}[H]
  \begin{minipage}{0.5\linewidth}
  \centerline{\includegraphics[width = \textwidth]{./fig/parameter1.png}}
  \end{minipage}
  \hfill
  \begin{minipage}{0.5\linewidth}
  \centerline{\includegraphics[width = \textwidth]{./fig/parameter2.png}}
  \end{minipage}
  \end{figure}
\end{itemize}


\section{结果展示}

左侧为输入图片的裁切,右侧为拟合曲线。

\begin{figure}[H]
\begin{minipage}{0.5\linewidth}
  \centerline{\includegraphics[width = 1.1\textwidth]{../figure/1_cut.png}}
\end{minipage}
\hfill
\begin{minipage}{0.5\linewidth}
  \centerline{\includegraphics[width = 1.1\textwidth]{../figure/1_fit.png}}
\end{minipage}
\caption{radian : 0.00784004 0.00778617 0.0560188 }
\end{figure}


\begin{figure}[H]
\begin{minipage}{0.48\linewidth}
  \centerline{\includegraphics[width = 1.1\textwidth]{../figure/2_cut.png}}
\end{minipage}
\hfill
\begin{minipage}{0.48\linewidth}
  \centerline{\includegraphics[width = 1.1\textwidth]{../figure/2_fit.png}}
\end{minipage}
\caption{radian : 0.12362332 0.0155812  0.00775662}
\end{figure}


\begin{figure}[H]
\begin{minipage}{0.48\linewidth}
  \centerline{\includegraphics[width = 1.1\textwidth]{../figure/3_cut.png}}
\end{minipage}
\hfill
\begin{minipage}{0.48\linewidth}
  \centerline{\includegraphics[width = 1.1\textwidth]{../figure/3_fit.png}}
\end{minipage}
\caption{radian : 0.00802192 0.60055564 0.0745157}
\end{figure}

\begin{figure}[H]
\begin{minipage}{0.48\linewidth}
  \centerline{\includegraphics[width = 1.1\textwidth]{../figure/4_cut.png}}
\end{minipage}
\hfill
\begin{minipage}{0.48\linewidth}
  \centerline{\includegraphics[width = 1.1\textwidth]{../figure/4_fit.png}}
\end{minipage}
\caption{radian : 0.12146684 0.01148749 0.00446624}
\end{figure}

 \begin{figure}[H]
\begin{minipage}{0.48\linewidth}
  \centerline{\includegraphics[width = 1.1\textwidth]{../figure/5_cut.png}}
\end{minipage}
\hfill
\begin{minipage}{0.48\linewidth}
  \centerline{\includegraphics[width = 1.1\textwidth]{../figure/5_fit.png}}
\end{minipage}
\caption{radian : 0.05094299 0.01116865 0.03528981}
\end{figure}

\begin{figure}[H]
\begin{minipage}{0.48\linewidth}
  \centerline{\includegraphics[width = 1.1\textwidth]{../figure/6_cut.png}}
\end{minipage}
\hfill
\begin{minipage}{0.48\linewidth}
  \centerline{\includegraphics[width = 1.1\textwidth]{../figure/6_fit.png}}
\end{minipage}
\caption{radian : 0.00536094 0.00430342 0.00902013}
\end{figure}

  \begin{figure}[H]
\begin{minipage}{0.48\linewidth}
  \centerline{\includegraphics[width = 1.1\textwidth]{../figure/7_cut.png}}
\end{minipage}
\hfill
\begin{minipage}{0.48\linewidth}
  \centerline{\includegraphics[width = 1.1\textwidth]{../figure/7_fit.png}}
\end{minipage}
\caption{radian : 0.10275823 0.0057146  0.00573458}
\end{figure}


 \begin{figure}[H]
\begin{minipage}{0.48\linewidth}
  \centerline{\includegraphics[width = 1.1\textwidth]{../figure/8_cut.png}}
\end{minipage}
\hfill
\begin{minipage}{0.48\linewidth}
  \centerline{\includegraphics[width = 1.1\textwidth]{../figure/8_fit.png}}
\end{minipage}
\caption{radian : 0.12433642 0.01011834 0.00875508}
\end{figure}

 \begin{figure}[H]
\begin{minipage}{0.48\linewidth}
  \centerline{\includegraphics[width = 1.1\textwidth]{../figure/9_cut.png}}
\end{minipage}
\hfill
\begin{minipage}{0.48\linewidth}
  \centerline{\includegraphics[width = 1.1\textwidth]{../figure/9_fit.png}}
\end{minipage}
\caption{radian : 0.00773065 0.00770052 0.01461635}
\end{figure}

 \begin{figure}[H]
\begin{minipage}{0.48\linewidth}
  \centerline{\includegraphics[width = 1.1\textwidth]{../figure/10_cut.png}}
\end{minipage}
\hfill
\begin{minipage}{0.48\linewidth}
  \centerline{\includegraphics[width = 1.1\textwidth]{../figure/10_fit.png}}
\end{minipage}
\caption{radian : 0.12203165 0.01043537 0.00586106}
\end{figure}

 \begin{figure}[H]
\begin{minipage}{0.48\linewidth}
  \centerline{\includegraphics[width = 1.1\textwidth]{../figure/11_cut.png}}
\end{minipage}
\hfill
\begin{minipage}{0.48\linewidth}
  \centerline{\includegraphics[width = 1.1\textwidth]{../figure/11_fit.png}}
\end{minipage}
\caption{radian : 0.00507701 0.02154793 0.00357411 0.01420405}
\end{figure}

 \begin{figure}[H]
\begin{minipage}{0.48\linewidth}
  \centerline{\includegraphics[width = 2\textwidth]{../figure/12_cut.png}}
\end{minipage}
\hfill
\begin{minipage}{0.48\linewidth}
  \centerline{\includegraphics[width = 2\textwidth]{../figure/12_fit.png}}
\end{minipage}
\caption{radian : 0.00552706 0.05701588}
\end{figure}


 \begin{figure}[H]
\begin{minipage}{0.48\linewidth}
  \centerline{\includegraphics[width = 1.1\textwidth]{../figure/13_cut.png}}
\end{minipage}
\hfill
\begin{minipage}{0.48\linewidth}
  \centerline{\includegraphics[width = 1.1\textwidth]{../figure/13_fit.png}}
\end{minipage}
\caption{radian : 0.00525744 0.0040048  0.01104831}
\end{figure}

\begin{figure}[H]
\begin{minipage}{0.48\linewidth}
  \centerline{\includegraphics[width = 1.1\textwidth]{../figure/14_cut.png}}
\end{minipage}
\hfill
\begin{minipage}{0.48\linewidth}
  \centerline{\includegraphics[width = 1.1\textwidth]{../figure/14_fit.png}}
\end{minipage}
\caption{radian : 0.00715935 0.27084455 0.62418447}
\end{figure}


\begin{figure}[H]
\begin{minipage}{0.48\linewidth}
  \centerline{\includegraphics[width = 1.1\textwidth]{../figure/15_cut.png}}
\end{minipage}
\hfill
\begin{minipage}{0.48\linewidth}
  \centerline{\includegraphics[width = 1.1\textwidth]{../figure/15_fit.png}}
\end{minipage}
\caption{radian: 0.00881595 0.01472718 0.00523674}
\end{figure}






\end{document}
