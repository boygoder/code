\documentclass[10pt]{beamer}
\usetheme{Madrid}

\usepackage{graphicx}
\usepackage{animate}
\usepackage{hyperref}
\usepackage{amsmath,bm,amsfonts,amssymb,enumerate,epsfig,bbm,calc,color,capt-of,multimedia,hyperref}
\usepackage{ctex}
\usepackage{fancybox}
\usepackage{times}
\usepackage{listings}
\usepackage{booktabs}
\usepackage{colortbl}

\usepackage{xcolor}
\definecolor{mygreen}{rgb}{0,0.6,0}
\definecolor{mygray}{rgb}{0.5,0.5,0.5}
\definecolor{mymauve}{rgb}{0.58,0,0.82}

\usepackage{fontspec} % 定制字体
\lstset{ %
	backgroundcolor=\color{white},      % choose the background color
	basicstyle=\footnotesize\ttfamily,  % size of fonts used for the code
	columns=fullflexible,
	tabsize=4,
	breaklines=true,               % automatic line breaking only at whitespace
	captionpos=b,                  % sets the caption-position to bottom
	commentstyle=\color{mygreen},  % comment style
	escapeinside={\%*}{*)},        % if you want to add LaTeX within your code
	keywordstyle=\color{blue},     % keyword style
	stringstyle=\color{mymauve}\ttfamily,  % string literal style
	frame=none,
	rulesepcolor=\color{red!20!green!20!blue!20},
	% identifierstyle=\color{red},
	language=c++,
}


\title{缆绳拟合中的问题与解决思路}
\subtitle{}
\author{史鸿儒}
\centering
\date{\today}


\begin{document}
\thispagestyle{empty}	%使封面没有导航条
\maketitle

%------------------目录页
	\begin{frame}           %生成目录页,目录太长时加选项[shrink]
%	\setcounter{page}{0}	%setcounter似乎对beamer无效
	\addtocounter{framenumber}{-2}%---------位置放在beginframe之后,不然无效
	\frametitle{目录}
	\thispagestyle{empty}
	\tableofcontents        % 也可以插入选项 [pausesections]
	%----------------------列目录时,隐藏所有的小节
	%\tableofcontents[hideallsubsections]
	\end{frame}




\section{悬链线}
\begin{frame}{悬链线}

\begin{block}{数据}
二值图的像素矩阵,白色像素处值为1,黑色像素处值为0。
\end{block}

\begin{block}{悬链线方程}
悬链线是一种曲线。一条理想的链条悬挂在两个支撑点之间,并受到均匀的重力作用,形成悬链线的形状。理想的链条是可以完美弯曲、不能拉伸并且整个密度相同的链条。支撑点可以处于不同的高度,形状仍然是悬链线。一般的悬链线方程为\(y = a \cosh((x-b)/a) + c\)
\end{block}
\begin{figure}
  \centerline{\includegraphics[height=3cm]{example.png}}
  \caption{$example$}
\end{figure}

\end{frame}

\section{拟合步骤}
\begin{frame}{1.单侧边缘提取}
单侧边缘提取:以水平方向为x轴,对于同一x,每条缆绳覆盖了多个像素点。通过单侧的边缘提取,每条缆绳的像素宽度变为1。
\begin{figure}
  \centerline{\includegraphics[height=3cm]{edge.png}}
  \caption{$edge$}
\end{figure}

\end{frame}

\begin{frame}{2.判定缆绳数量}
判定缆绳数量:对经过边缘提取后的像素矩阵,每一列求和,并消除可能出现的干扰点影响,最终求和结果的最大值就是图片中的缆绳数量。
\begin{figure}
  \centerline{\includegraphics[height=3cm]{cut.png}}
  \caption{$cut$}
\end{figure}
\end{frame}

\begin{frame}{3.区分不同缆绳}
区分不同缆绳:首先取出一段不同缆绳的无交点部分作为样本点,进行第一次曲线拟合。
\begin{figure}
  \centerline{\includegraphics[height=3cm]{first_fit.png}}
  \caption{$first_fit$}
\end{figure}
\end{frame}


\begin{frame}{4.曲线拟合}
曲线拟合:为了获取周围的样本点,根据拟合出的近似曲线,依次计算样本点和每一条曲线的距离。根据距离大小决定样本点应该归属的曲线,进行下一次曲线拟合。重复,直至获取所有的样本点。
\begin{figure}
  \centerline{\includegraphics[height=3cm]{final_fit.png}}
  \caption{$final_fit$}
\end{figure}
\end{frame}

\begin{frame}{5.计算弧度}
计算弧度:弧度定义:每条曲线2个顶点连成一条直线,直线长度L,计算曲线上每个点和直线的距离,最大距离为H,弧度定义为H/L。这里两个顶点取为拟合曲线所用到的最左侧和最右侧样本点。
\begin{figure}
  \centerline{\includegraphics[height=3cm]{radian.png}}
  \caption{$radian$}
\end{figure}
\end{frame}


\section{拟合问题}
\begin{frame}{交点部分过长}
\begin{figure}[H]
\begin{minipage}{0.48\linewidth}
  \centerline{\includegraphics[width=6cm]{bad_example1.png}}
  \caption{原图}
\end{minipage}
\hfill
\begin{minipage}{0.48\linewidth}
  \centerline{\includegraphics[width=6cm]{bad_fit1.png}}
  \caption{交点部分过长时}
\end{minipage}
\end{figure}

原因:可供初始拟合的样本点(无交点部分)过短。
\end{frame}

\begin{frame}{交点部分过短}

\begin{figure}[H]
\begin{minipage}{0.48\linewidth}
  \centerline{\includegraphics[width=6cm]{bad_example2.png}}
  \caption{原图}
\end{minipage}
\hfill
\begin{minipage}{0.48\linewidth}
  \centerline{\includegraphics[width=6cm]{bad_fit2.png}}
  \caption{交点部分过短时}
\end{minipage}

原因:无法将其识别为两条曲线。
\end{figure}
\end{frame}


\section{断点连接}

\begin{frame}{断点连接}
数据:二值图的像素矩阵,白色像素处值为1,黑色像素处值为0。
要求:将下图中原本是一条缆绳的部分,连接起来。
\begin{figure}
  \centerline{\includegraphics[height=5cm]{breakpoint.png}}
  \caption{例图}
\end{figure}

\end{frame}


\begin{frame}{解决思路}
\begin{enumerate}
\item 识别端点:通过遍历白色点的左上邻域,查看是否存在白色点,来确定该点是否为左端点。同理,遍历右下邻域来确定该点是否为右端点。
\item 将要连接的端点组成点对:计算所有左端点与右端点之间的距离,将距离最小的点对连接。
\item 删除已经连接的点对。
\item 重复第2、3步,直到点对之间的最小距离大于阈值。
\end{enumerate}
\begin{figure}
  \centerline{\includegraphics[height=3cm]{neighbors.png}}
  \caption{八邻域}
\end{figure}
\end{frame}

\begin{frame}{测试}
\begin{figure}[H]
\begin{minipage}{0.48\linewidth}
  \centerline{\includegraphics[width=5cm]{breakpoint.png}}
  \caption{原图}
\end{minipage}
\hfill
\begin{minipage}{0.48\linewidth}
  \centerline{\includegraphics[width=6cm]{connect.png}}
  \caption{断点连接}
\end{minipage}
\end{figure}
\end{frame}

\begin{frame}{存在的问题}
当断点之间距离过远时,会错连。
\begin{figure}[H]
\begin{minipage}{0.48\linewidth}
  \centerline{\includegraphics[width=5cm]{bad_breakpoint.png}}
  \caption{原图}
\end{minipage}
\hfill
\begin{minipage}{0.48\linewidth}
  \centerline{\includegraphics[width=6cm]{bad_connect.png}}
  \caption{断点连接}
\end{minipage}
\end{figure}

\end{frame}
\subsection{$output\_results( )$}
%\begin{frame}{结果示例}
%\begin{figure}[H]
%\begin{minipage}{0.48\linewidth}
%  \centerline{\includegraphics[height=4.8cm]{vtk0.png}}
%  \caption{边界条件为$u = 0$.}
%\end{minipage}
%\hfill
%\begin{minipage}{0.48\linewidth}
%  \centerline{\includegraphics[height=4.8cm]{vtk1.png}}
%  \caption{$y=-1$和$y=1$边界条件为$u=1$,其余为$u=0$.}
%\end{minipage}
%\end{figure}
%\end{frame}



\begin{frame}
\huge{\centerline{The End}}
\end{frame}

\end{document}

