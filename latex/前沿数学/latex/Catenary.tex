\documentclass[10.5pt]{beamer}
\usetheme{Madrid}

\usepackage{graphicx}
\usepackage{animate}
\usepackage{hyperref}
\usepackage{amsmath,bm,amsfonts,amssymb,enumerate,epsfig,bbm,calc,color,capt-of,multimedia,hyperref}
\usepackage{ctex}
\usepackage{fancybox}
\usepackage{times}
\usepackage{listings}
\usepackage{booktabs}
\usepackage{colortbl}
\usepackage{float}
\usepackage{xcolor}
\definecolor{mygreen}{rgb}{0,0.6,0}
\definecolor{mygray}{rgb}{0.5,0.5,0.5}
\definecolor{mymauve}{rgb}{0.58,0,0.82}
\usepackage{fontspec} % 定制字体
\setlength{\parindent}{2em}

\lstset{ %
	backgroundcolor=\color{white},      % choose the background color
	basicstyle=\footnotesize\ttfamily,  % size of fonts used for the code
	columns=fullflexible,
	tabsize=4,
	breaklines=true,               % automatic line breaking only at whitespace
	captionpos=b,                  % sets the caption-position to bottom
	commentstyle=\color{mygreen},  % comment style
	escapeinside={\%*}{*)},        % if you want to add LaTeX within your code
	keywordstyle=\color{blue},     % keyword style
	stringstyle=\color{mymauve}\ttfamily,  % string literal style
	frame=none,
	rulesepcolor=\color{red!20!green!20!blue!20},
	% identifierstyle=\color{red},
	language=c++,
}


\title{图像处理与网格简化}
\subtitle{}
\author{史鸿儒}
\date{2022年12月30日}


\begin{document}
\thispagestyle{empty}	%使封面没有导航条
\maketitle

%------------------目录页
	\begin{frame}           %生成目录页,目录太长时加选项[shrink]
%	\setcounter{page}{0}	%setcounter似乎对beamer无效
	\addtocounter{framenumber}{-2}%---------位置放在beginframe之后,不然无效
	\frametitle{目录}
	\thispagestyle{empty}
	\tableofcontents        % 也可以插入选项 [pausesections]
	%----------------------列目录时,隐藏所有的小节
	%\tableofcontents[hideallsubsections]
	\end{frame}




\section{颜色转换}
\begin{frame}{颜色转换}
\begin{figure}[!t]
\centering
\includegraphics[width=3.5in]{fig/ColorTransfer.png}
\caption{图像渲染}
\label{fig_1}
\end{figure}
(a)图像提供要处理的图像,称为源图像。
(b)图像提供颜色,称为目标图像。

\end{frame}

\begin{frame}{颜色空间}
核心策略是选择一个合适的颜色空间,在其中进行均值与标准差变换。
当图像使用经典的RGB三通道颜色空间表示时,不同通道的值之间存在相关性。


$l \alpha \beta$颜色空间:$l$分量密切匹配人类亮度感知,可以通过修改$\alpha$和$\beta$分量的输出色阶来做精确的颜色平衡,使用l分量来调整亮度对比。直观上,三个通道分别代表:亮度,黄蓝通道,红绿通道。

\end{frame}

\begin{frame}{均值与方差变换}
将图片像素从RGB颜色空间变换到$l \alpha \beta$颜色空间。
在$l \alpha \beta$颜色空间中,做均值与反差变换:
\begin{enumerate}
\item 将源图像每一像素的颜色减去源图像的颜色均值。
    $l^{*} = l - \hat{l}_{s},\alpha^{*} = \alpha - \hat{\alpha}_{s},\beta^{*} = \beta  - \hat{\beta}_{s}$
    其中,$\hat{l}_{s}$、$\hat{\alpha}_{s}$、$\hat{\beta}_{s}$代表源图像三个颜色通道的均值。
    
\item 乘以目标图像的标准差与源图像的标准差的比例。
    $l^{'} = \dfrac{\sigma_{t}^{l}}{\sigma_{s}^{l} }l^{*},\alpha^{'} = \dfrac{\sigma_{t}^{\alpha}}{\sigma_{s}^{\alpha}} \alpha^{*},\beta^{'} = \dfrac{\sigma_{t}^{\beta}}{\sigma_{s}^{\beta} }\beta^{*}$

    其中,源图像的颜色标准差为$\sigma_{s}$,目标图像的颜色标准差为$\sigma_{t}$。
\item  将源图像每一像素的颜色加上目标图像的颜色均值。
    $l = l^{'} + \hat{l}_{t},\alpha = \alpha^{'} + \hat{\alpha}_{t},\beta = \beta^{'}  + \hat{\beta}_{t}$

  其中,$\hat{l}_{t}$、$\hat{\alpha}_{t}$、$\hat{\beta}_{t}$代表目标图像三个颜色通道的均值。
\end{enumerate}

最后,变换回RGB颜色空间。

\end{frame}



\section{图像扭转}
\begin{frame}{图像扭转}
\begin{figure}[!t]
\centering
\includegraphics[width=4in]{fig/ImageWarping.png}
\caption{图像扭转}
\label{fig_2}
\end{figure}
\end{frame}

\begin{frame}{算法描述}
\begin{block}{}
输入:n对锚点$(p_{i},q_{i}),p_{i}= (x_{i},y_{i}),q_{i} = (x^{'}_{i},y^{'}_{i}),i = 1,\cdots,n$。

输出: 连续映射$f: \mathbf{R}^{2} \rightarrow \mathbf{R}^{2}$,满足$f(p_{i}) = q_{i},i = 1,\cdots,n$。

将映射f作用在所有像素点上,并对产生的缝隙进行填补。
\end{block}

\begin{figure}[!t]
\centering
\includegraphics[width=2in]{fig/AnchorPoints.png}
\caption{锚点}
\label{fig_3}
\end{figure}
\end{frame}


\begin{frame}{IDW算法}
 IDW 为反距离加权插值(Inverse Distance Weighted)。
 对于每对锚点$(p_{i},q_{i}),p_{i}= (x_{i},y_{i}),q_{i} = (x^{'}_{i})$,寻找局部近似函数$f_{i,x},f_{i,y}:\mathbb{R}^{2} \rightarrow \mathbb{R}$,使得$f_{i,x}(p_{i}) = x_{i}^{'}$,$f_{i,y}(p_{i}) = y_{i}^{'}$。
 
 坐标变换公式为:
 $$f(p) = \sum\limits_{i=1}^{n} \omega_{i}(p) (f_{i,x}(p),f_{i,y}(p))$$
 其中,$\omega_{i}(p)$是权重函数,满足:
 \begin{equation}\nonumber
 \left\{
 \begin{aligned}
 & w_{i}(p_{i}) = 1 \\
 & \sum\limits_{j=1}^{n} w_{i}(p) = 1,\forall p \in \mathbb{R}^{2} \\
 & w_{i}(p) > 0, i = 1,\cdots,n
 \end{aligned}
  \right.
 \end{equation}
 权重函数$\omega_{i}(p) = \dfrac{\sigma_{i}(p)}{\sum_{j=1}^{n} \sigma_{j}(p)}$,$\sigma
_{i}(p) = \dfrac{1}{d(p,p_{i})}$。

\end{frame}

\begin{frame}{IDW算法}
  
取局部近似函数为线性函数,比如$f_{i,x}(p) = x_{i}^{'} + (p - p_{i})T_{i}$。

$T_{i}$是$2 \times 1$的矩阵,选取$T_{i}$使误差函数最小。
\begin{equation}\nonumber
\begin{aligned}
E(T_{i})
&= \sum\limits_{j=1,j\neq i}^{n} \sigma_{i}(p_{j}) (f_{i,x}(p_{j}) - x_{j}^{'})^{2}  \\
&= \sum\limits_{j=1,j \neq i}^{n} \sigma_{i}(p_{j}) (x_{i}^{'} + (p_{j} - p_{i})T_{i} - x_{j}^{'})^{2}
\end{aligned}
\end{equation}
\end{frame}



\begin{frame}{RBF算法}
RBF为径向基函数(Radial Basis Function)。

坐标变换公式为$f(p) = \sum\limits_{i=1}^{n} \alpha_{i} g_{i}(d(p,p_{i})) + Ap + B$。
为简单起见,取变换公式为$f(p) = \sum\limits_{i=1}^{n} \alpha_{i} g_{i}(d(p,p_{i})) + p$。

径向基函数取Hardy multiquadrics:
$$g_{i}(d) = (d^2 + r_{i}^{2})^{1/2},r_{i} = \min_{j \neq i}{d(p_{i},p_{j})}$$

权重$\alpha_{i}$通过求解$f(p_{i}) = q_{i}$得到。

\end{frame}


\begin{frame}{效果展示}

\begin{figure}[H]
\begin{minipage}{0.48\linewidth}
  \centerline{\includegraphics[width=3cm]{fig/IDW_UP.png}}
  \caption{IDW:嘴角上扬}
\end{minipage}
\hfill
\begin{minipage}{0.48\linewidth}
  \centerline{\includegraphics[width=3cm]{fig/IDW_DOWN.png}}
  \caption{IDW:嘴角下垂}
\end{minipage}
\vfill
\begin{minipage}{0.48\linewidth}
  \centerline{\includegraphics[width=3cm]{fig/RBF_UP.png}}
  \caption{RBF:嘴角上扬}
\end{minipage}
\hfill
\begin{minipage}{0.48\linewidth}
  \centerline{\includegraphics[width=3cm]{fig/RBF_DOWN.png}}
  \caption{RBF:嘴角下垂}
\end{minipage}
\end{figure}

\end{frame}

\section{网格简化}
\begin{frame}{网格简化}
3D模型的表面可以由三角网格表示,网格越精细,模型越复杂。为了兼顾渲染效率和渲染质量,当模型在远处或快速移动时,渲染质量的好坏对观感的影响比较小,可以降低渲染精度。
这就涉及到使用网格简化算法来在不同情况下生成不同细节层次(Level Of Detail, LOD)的模型,LOD技术主要广泛用于沙盒游戏或开放世界游戏。

\begin{figure}[!t]
\centering
\includegraphics[width=5in]{fig/mesh_simplication.png}
\caption{网格简化}
\label{fig_4}
\end{figure}
\end{frame}


\begin{frame}{QEM网格简化算法}
QEM(Quadratic Error Metrics)是一种网格简化算法,算法步骤如下:
\begin{enumerate}
\item 对所有初始的顶点计算Q矩阵。
\item 取出所有的有效点对。
\item 对每对有效点对,计算最优收缩点$v^{*}$,以及收缩有效点对的代价。
\item 依次收缩代价最小的点对,直至模型满足要求。
\end{enumerate}
\end{frame}

\begin{frame}{1.对所有初始的顶点计算Q矩阵}
每个顶点都是一组三角形的交点,每个三角形确定了一张平面。 我们将每个顶点与一组平面相关联,将顶点相对于该组平面的误差定义为顶点与平面距离的平方和。对于每个顶点$v = (v_{x},v_{y},v_{z})^{T}$,定义$\overline{v} = (v_{x},v_{y},v_{z},1)^{T}$,误差公式如下:
\begin{equation}
\begin{aligned}
\Delta(v) 
&= \sum_{p \in planes(v)} (p^{T}\overline{v})^{2} \\
&= \sum_{p \in planes(v)} \overline{v}^{T} (p p^{T}) \overline{v} \\
&= \overline{v}^{T} (\sum_{p \in planes(v)} \mathbf{K}_{p}) \overline{v}
\end{aligned}
\end{equation}

其中$\mathbf{p} = [a,b,c,d]^{T}$,方程$ax+by+cz+d = 0$确定了一张平面,$a^2 + b^2 + c^2  = 1$。

$\mathbf{K}_{p} = p p^{T}$,Q矩阵的定义为$Q = \sum_{p \in planes(v)} \mathbf{K}_{p}$。

\end{frame}


\begin{frame}{2.取出所有的有效点对}
我们说一对点$(v_{1},v_{2})$是有效对,如果:
\begin{enumerate}
\item $(v_{1},v_{2})$是一条边的顶点。
\item 或者$||v_{1} - v_{2} || < t$,t为选定的阈值。
\end{enumerate}

第二个条件是处理不连通的模型,收缩本没有边但距离很近的点。如果没有第二个条件,那么在简化网格时,本来相邻很近的点会被拉远,此时可能会形成本不该有的空洞。

\end{frame}

\begin{frame}{3.对每对有效点对,计算最优收缩点$v^{*}$,以及收缩有效点对的代价。}

当我们将$(v_{1},v_{2})$收缩为$v^{*}$后,我们定义$planes(v^{*}) = planes(v_{1}) \cup planes(v_{2})$,若$Q_{1} = \sum_{p \in planes(v1)} \mathbf{K}_{p}$,$Q_{2} = \sum_{p \in planes(v2)} \mathbf{K}_{p}$,那么$Q^{*} = Q_{1} + Q_{2}$。

收缩有效对的代价定义为$\Delta(v^{*}) = \overline{v^{*}}^{T} Q^{*} \overline{v^{*}} =  \overline{v^{*}}^{T} (Q_{1} + Q_{2}) \overline{v^{*}}$。

最优收缩点就是使得收缩代价最小的点,即
$$v^{*}= \mathop{\arg\min}\limits_{v} \overline{v}^{T} (Q_{1} + Q_{2}) \overline{v}$$


\end{frame}

\begin{frame}{4.依次收缩代价最小的点对,直至模型满足要求}
\begin{figure}[H]
\begin{minipage}{0.48\linewidth}
  \centerline{\includegraphics[width=4cm]{fig/cow5804.png}}
  \caption{三角形数:5804}
\end{minipage}
\hfill
\begin{minipage}{0.48\linewidth}
  \centerline{\includegraphics[width=4cm]{fig/cow994.png}}
  \caption{三角形数:994}
\end{minipage}
\vfill
\begin{minipage}{0.48\linewidth}
  \centerline{\includegraphics[width=4cm]{fig/cow532.png}}
  \caption{三角形数:532}
\end{minipage}
\hfill
\begin{minipage}{0.48\linewidth}
  \centerline{\includegraphics[width=4cm]{fig/cow300.png}}
  \caption{三角形数:300}
\end{minipage}
\end{figure}

\end{frame}

\section{边缘提取}

\begin{frame}{Sobel算子}
Sobel算子是一种用于边缘检测的离散微分算子。Sobel算子计算像素点上下、左右邻点灰度的加权差,根据在边缘处达到极值这一现象检测边缘。对噪声具有平滑作用,提供较为精确的边缘方向信息。

Sobel卷积因子为:
\begin{figure}[!t]
\centering
\includegraphics[width=2.8in]{fig/sobel_factor.png}
\caption{Sobel卷积因子}
\label{fig_4}
\end{figure}
将卷积因子与图像做平面卷积,通过$G = \sqrt{G_{x}^{2} + G_{y}^{2}}$计算像素梯度,将梯度值超过阈值的点标记为边缘点。

    
\end{frame}

\begin{frame}{Sobel边缘提取}
\begin{figure}[!t]
\centering
\includegraphics[width=4in]{fig/sobel_local.png}
\caption{Sobel边缘提取}
\label{fig_5}
\end{figure}

Sobel算子对边缘定位不是很准确,图像的边缘不止一个像素,当对精度要求不是很高时,可以使用。
\end{frame}


\begin{frame}{Canny边缘检测}
Canny边缘检测算法可以分为以下5个步骤:
\begin{enumerate}
\item 使用高斯滤波器,以平滑图像,滤除噪声。
	大小为$(2k+1) \times (2k+1)$的高斯滤波器核的生成方程式为:
	$$H_{ij} = \frac{1}{2 \pi \sigma} e ^ {-\frac{(i-(k+1))^{2} + (j-(k+1))^{2}}{2 \sigma^{2}}}; 1 \leq i,j \leq (2k+1)$$
	比如当$\sigma = 1.4$时,$3\times 3$的高斯滤波器核为:
	$$
	H = 
	\begin{bmatrix}
	0.0924 & 0.1192 & 0.0924\\
	0.1192 & 0.1538 & 0.1192\\
	0.0924 & 0.1192 & 0.0924\\	
	\end{bmatrix}
	$$
	高斯卷积核大小的选择将影响Canny检测器的性能。卷积核越大,检测器对噪声的敏感度越低,但是边缘检测的定位误差也将略有增加。
\item 计算图像中每个像素点的梯度强度和方向。
	与Sobel算子类似,$G = \sqrt{G_{x}^{2} + G_{y}^{2}}$,$\theta = arctan(G_{y}/G_{x})$。
\end{enumerate}

\end{frame}

\begin{frame}{Canny边缘检测}
\begin{enumerate}
\setcounter{enumi}{2}
\item 应用非极大值抑制,以将局部最大值之外的梯度值抑制为0。
	\begin{itemize}
	\item 将当前像素的梯度强度与沿正负梯度方向上的两个像素进行比较。
	\item  如果当前像素的梯度强度最大,则该像素点保留为边缘点,否则该像素点将被抑制。
	\end{itemize}
\item 应用双阈值检测来确定真实的和潜在的边缘。
	\begin{itemize}
	\item 梯度值高于高阈值,标记为强边缘。
	\item 梯度值小于高阈值并且大于低阈值,标记为弱边缘。
	\item 梯度值小于低阈值,则会被抑制。
	\end{itemize}

\item 通过抑制孤立的弱边缘最终完成边缘检测。
	
	强边缘已经被确定为边缘。对于弱边缘,如果与强边缘连接,那么归为边缘;否则,弱边缘会被抑制。
\end{enumerate}

\end{frame}

\begin{frame}{Canny边缘检测}
\begin{figure}[!t]
\centering
\includegraphics[width=4in]{fig/canny_local.png}
\caption{Canny边缘提取}
\label{fig_6}
\end{figure}
Canny方法不容易受噪声干扰,能够检测到真正的弱边缘,做到边缘的像素宽度为1。
\end{frame}

\begin{frame}{Laplacian算子}
Laplacian算子是二阶微分算子。
\begin{equation}\nonumber
\begin{aligned}
&\Delta f = \frac{\partial^{2} f}{\partial x^2} + \frac{\partial^{2} f}{\partial y^2} \\
&\frac{\partial^{2} f}{\partial x^2} = f(x+1,y) - 2f(x,y) + f(x-1,y) \\
&\frac{\partial^{2} f}{\partial y^2} = f(x,y+1) - 2f(x,y) + f(x,y-1) 
\end{aligned}
\end{equation}
写作卷积模板:
$$
\Delta f \approx
\begin{bmatrix}
0 & 1 & 0 \\
1 & -4 & 1 \\
0 & 1 & 0
\end{bmatrix}
$$
\end{frame}

\begin{frame}{Laplacian算子}
\begin{figure}[!t]
\centering
\includegraphics[width=4in]{fig/laplace_local.png}
\caption{Laplacian边缘提取}
\label{fig_7}
\end{figure}

Laplacian对于噪声比较敏感,会产生不连续的边缘,很少用于边缘检测。可增强图像中灰度突变的区域,减弱灰度的缓慢变化区域,从而实现图像锐化。

\end{frame}
\begin{frame}
\huge{\centerline{The End}}
\end{frame}

\end{document}

